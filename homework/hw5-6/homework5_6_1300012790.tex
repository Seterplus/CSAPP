\documentclass {article}
\usepackage {geometry}
\usepackage {CJK}
\usepackage {amsmath}
\usepackage {amssymb}
\usepackage {indentfirst}
\usepackage {listings}
\usepackage {courier}

\lstset{basicstyle=\ttfamily,breaklines=true,numbers=left}

\begin{document}
  \begin {CJK*} {UTF8} {gbsn}
    \title {\textbf {\Huge Homework 5 \& Homework 6}}
		\author {郭天魁 \\ 信息科学技术学院 \\ 1300012790}

		\maketitle

		\section{Homework 5}
			\subsection{3.61}
				\begin{lstlisting}[language=C]
int fix_var_prod_ele(int n, int A[n][n], int B[n][n], int i, int k) {
	register int *Aptr = A[i];
	register int *Bptr = &B[n - 1][k];
	register int j = n;
	register int result = 0;
	register int N = n;
	while(--j != -1) {
		result += Aptr[j] * (*Bptr);
		Bptr -= N;
	}
	return result;
}
				\end{lstlisting}

				编译后,循环体中指令如下:
				\begin{lstlisting}[language={[x86masm]Assembler}]
.L3:
	movl	(%esi,%edx,4), %ebx
	imull	(%ecx), %ebx
	addl	%ebx, %eax
	addl	%edi, %ecx
	subl	$1, %edx
	cmpl	$-1, %edx
	jne	.L3
				\end{lstlisting}

				将循环变量$j$由从$0$到$n-1$循环替换为从$n-1$到$0$循环,可以避免在循环体中使用值$n$。\\

				需要注意的是必须使用gcc的-O1编译选项。如果使用-O0,则imull指令中会用到\%edx和\%eax两个临时寄存器变量,导致使用的五个值至少有一个需要被保存在内存中。\\

			\subsection{3.62}
				A. $M=13.$\\

				B. i: \%edi, j: \%ecx.\\

				C.
				\begin{lstlisting}[language=C]
void transpose(int M, int A[M][M]) {
	int i, j;
	int M4 = M << 2;
	int eax, *ebx, *esi, edx;
	for (i = 0; i < M; i++) {
		ebx = &A[0][i];
		esi = A[i];
		for (j = 0; j < i; j++) {
			eax = *ebx;
			edx = esi[j];
			esi[j] = eax;
			*ebx = edx;
			ebx = (void*)ebx + M4;
		}
	}
}
				\end{lstlisting}
			
			\subsection{3.64}
				A. 8(\%ebp)是result的地址,12(\%ebp)是s1.a,16(\%ebp)是s1.p.\\

				B. 从高到低依次为s2.diff, s2.sum, s1.p, s1.a, \&s2.\\

				C. 将参数结构体内的变量由低到高存入栈中。\\

				D. 将返回值的结构体地址压栈,在函数中直接操作结构体。\\

		\section{Homework 6}
			\subsection{3.67}
				A. e1.p: 0, e1.y: 4, e2.x: 0, e2.next: 4.\\

				B. 8.\\

				C.
				\begin{lstlisting}[language=C]
up->e2.next->e1.y=*(up->e2.next->e1.p)-up->e2.x;
				\end{lstlisting}

			\subsection{3.68}
				\begin{lstlisting}[language=C]
#define L 10

void good_echo(){
	char buf[L];
	char *p;
	while(fgets(buf, L, stdin)) {
		for(p = buf; *p; p++) {
			putchar(*p);
			if(*p == '\n') return;
		}
	}
}
				\end{lstlisting}

			\subsection{3.70}
				A.
				\begin{lstlisting}[language=C]
long traverse(tree_ptr tp) {
	long rax = 1LL << 63;
	long rbx, r12;
	if(tp) {
		rbx = tp->val;
		r12 = traverse(tp->left);
		rax = traverse(tp->right);
		rax = r12 >= rax ? r12 : rax;
		rax = rax < rbx ? rbx : rax;
	}
	return rax;
}
				\end{lstlisting}

				B. Find the maximum value in the binary tree whose root is tp.\\

	\end {CJK*}
\end {document}

