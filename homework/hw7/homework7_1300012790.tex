\documentclass {article}
\usepackage {geometry}
\usepackage {CJK}
\usepackage {amsmath}
\usepackage {amssymb}
\usepackage {indentfirst}
\usepackage {listings}
\usepackage {courier}

\lstset{basicstyle=\ttfamily,breaklines=true,numbers=left}

\lstdefinelanguage
	[Y86]{Assembler}
	[x86masm]{Assembler}
	{morekeywords={irmovl,pushl,popl,addl,subl,rrmovl,rmmovl,halt,mrmovl,andl,xorl,cmovl,.long,.pos}}

\begin{document}
  \begin {CJK*} {UTF8} {gbsn}
    \title {\textbf {\Huge Homework 7}}
		\author {郭天魁 \\ 信息科学技术学院 \\ 1300012790}

		\maketitle

	 \section{Homework 7}
			\subsection{4.43}
				A. 这段代码将push执行前\%esp$-4$的值,且会修改CFLAGS。\\

				B.
				\begin{lstlisting}[language={[x86masm]Assembler}]
movl REG,-4(%esp)
leal -4(%esp),%esp
				\end{lstlisting}

			\subsection{4.45}
				A.
				\begin{lstlisting}[language=C]
void bubble_p(int *data, int count) {
	int i, last;
	int *p1, *p2;
	for (last = count-1; last > 0; last--) {
		for (i = 0; i < last; i++) {
			p1 = data + i;
			p2 = p1 + 1;
			if (*p2 < *p1) {
				/* Swap adjacent elements */
				int t = *p2;
				*p2 = *p1;
				*p1 = t;
			}
		}
	}
}

int a[] = {4,3,9,1,0};

int main() {
	int i;
	bubble_p(a, 5);
	for(i = 0; i < 4; i++)
		if(a[i] > a[i+1])
			return 1;
	return 0;
}
				\end{lstlisting}

				B.
				\begin{lstlisting}[language={[Y86]Assembler}]
	.pos 0
init:	irmovl Stack, %esp
	irmovl Stack, %ebp
	call Main
	halt

bubble_p:
	pushl	%edi
	pushl	%esi
	pushl	%ebx
	mrmovl	20(%esp), %edi
	irmovl	$1, %eax
	rrmovl	%edi, %esi
	subl	%eax, %esi
	mrmovl	16(%esp), %eax
	andl	%esi, %esi
	jle	L1
	rrmovl	%eax, %edi
	rrmovl	%esi, %ebx
	addl	%ebx, %ebx
	addl	%ebx, %ebx
	addl	%eax, %ebx
	jmp	L3
L7:
	mrmovl	4(%eax), %edx
	mrmovl	(%eax), %ecx
	pushl	%edx
	subl	%ecx, %edx
	popl	%edx
	jge	L4
	rmmovl	%ecx, 4(%eax)
	rmmovl	%edx, (%eax)
L4:
	pushl	%ebx
	irmovl	$4, %ebx
	addl	%ebx, %eax
	popl	%ebx
	pushl	%eax
	subl	%ebx, %eax
	popl	%eax
	jne	L7
L6:
	pushl	%eax
	irmovl	$4, %eax
	subl	%eax, %ebx
	irmovl	$1, %eax
	subl	%eax, %esi
	popl	%eax
	je	L1
L3:
	andl	%esi, %esi
	jle	L6
	rrmovl	%edi, %eax
	jmp	L7
L1:
	popl	%ebx
	popl	%esi
	popl	%edi
	ret
Main:
	irmovl	$8, %eax
	subl	%eax, %esp
	irmovl	$5, %eax
	rmmovl	%eax, 4(%esp)
	irmovl	$a, %eax
	rmmovl	%eax, (%esp)
	call	bubble_p
	irmovl	$8, %eax
	addl	%eax, %esp
	xorl	%eax, %eax
	ret

	.align 4
a:
	.long	4
	.long	3
	.long	9
	.long	1
	.long	0

	.pos 0x1000
Stack:
				\end{lstlisting}

				输出为:
				\begin{lstlisting}
Stopped in 235 steps at PC = 0x11.  Status 'HLT', CC Z=1 S=0 O=0

Changes to registers:
%ecx:	0x00000000	0x00000001
%esp:	0x00000000	0x00001000
%ebp:	0x00000000	0x00001000

Changes to memory:
0x00dc:	0x00000004	0x00000000
0x00e0:	0x00000003	0x00000001
0x00e4:	0x00000009	0x00000003
0x00e8:	0x00000001	0x00000004
0x00ec:	0x00000000	0x00000009
0x0fe0:	0x00000000	0x000000e0
0x0ff0:	0x00000000	0x000000d1
0x0ff4:	0x00000000	0x000000dc
0x0ff8:	0x00000000	0x00000005
0x0ffc:	0x00000000	0x00000011
				\end{lstlisting}

				可以看出内存中的值已被排序。\\

			\subsection{4.46}
				\begin{lstlisting}[language={[Y86]Assembler}]
	.pos 0
init:
	irmovl Stack, %esp
	irmovl Stack, %ebp
	call Main
	halt

bubble_p:
	pushl	%edi
	pushl	%esi
	pushl	%ebx
	mrmovl	20(%esp), %edi
	irmovl	$1, %eax
	rrmovl	%edi, %esi
	subl	%eax, %esi
	mrmovl	16(%esp), %eax
	andl	%esi, %esi
	jle	L1
	rrmovl	%eax, %edi
	rrmovl	%esi, %ebx
	addl	%ebx, %ebx
	addl	%ebx, %ebx
	addl	%eax, %ebx
	jmp	L3
L7:
	mrmovl	4(%eax), %edx
	mrmovl	(%eax), %ecx
	pushl	%ebx
	pushl	%edx
	subl	%ecx, %edx
	popl	%edx
	cmovl	%ecx, %ebx
	cmovl	%edx, %ecx
	cmovl	%ebx, %edx
	rmmovl	%edx, 4(%eax)
	rmmovl	%ecx, (%eax)
	irmovl	$4, %ebx
	addl	%ebx, %eax
	popl	%ebx
	pushl	%eax
	subl	%ebx, %eax
	popl	%eax
	jne	L7
L6:
	pushl	%eax
	irmovl	$4, %eax
	subl	%eax, %ebx
	irmovl	$1, %eax
	subl	%eax, %esi
	popl	%eax
	je	L1
L3:
	andl	%esi, %esi
	jle	L6
	rrmovl	%edi, %eax
	jmp	L7
L1:
	popl	%ebx
	popl	%esi
	popl	%edi
	ret
Main:
	irmovl	$8, %eax
	subl	%eax, %esp
	irmovl	$5, %eax
	rmmovl	%eax, 4(%esp)
	irmovl	$a, %eax
	rmmovl	%eax, (%esp)
	call	bubble_p
	irmovl	$8, %eax
	addl	%eax, %esp
	xorl	%eax, %eax
	ret

	.align 4
a:
	.long	4
	.long	3
	.long	9
	.long	1
	.long	0

	.pos 0x1000
Stack:
				\end{lstlisting}

			\subsection{4.49}
				在实验材料中的适当位置添加IIADDL即可。
				\begin{lstlisting}
#/* $begin seq-all-hcl */

####################################################################

#  HCL Description of Control for Single Cycle Y86 Processor SEQ   #

#  Copyright (C) Randal E. Bryant, David R. O'Hallaron, 2010       #

####################################################################



## Your task is to implement the iaddl and leave instructions

## The file contains a declaration of the icodes

## for iaddl (IIADDL) and leave (ILEAVE).

## Your job is to add the rest of the logic to make it work



####################################################################

#    C Include's.  Don't alter these                               #

####################################################################



quote '#include <stdio.h>'

quote '#include "isa.h"'

quote '#include "sim.h"'

quote 'int sim_main(int argc, char *argv[]);'

quote 'int gen_pc(){return 0;}'

quote 'int main(int argc, char *argv[])'

quote '  {plusmode=0;return sim_main(argc,argv);}'



####################################################################

#    Declarations.  Do not change/remove/delete any of these       #

####################################################################



##### Symbolic representation of Y86 Instruction Codes #############

intsig INOP 	'I_NOP'

intsig IHALT	'I_HALT'

intsig IRRMOVL	'I_RRMOVL'

intsig IIRMOVL	'I_IRMOVL'

intsig IRMMOVL	'I_RMMOVL'

intsig IMRMOVL	'I_MRMOVL'

intsig IOPL	'I_ALU'

intsig IJXX	'I_JMP'

intsig ICALL	'I_CALL'

intsig IRET	'I_RET'

intsig IPUSHL	'I_PUSHL'

intsig IPOPL	'I_POPL'

# Instruction code for iaddl instruction

intsig IIADDL	'I_IADDL'

# Instruction code for leave instruction

intsig ILEAVE	'I_LEAVE'



##### Symbolic represenations of Y86 function codes                  #####

intsig FNONE    'F_NONE'        # Default function code



##### Symbolic representation of Y86 Registers referenced explicitly #####

intsig RESP     'REG_ESP'    	# Stack Pointer

intsig REBP     'REG_EBP'    	# Frame Pointer

intsig RNONE    'REG_NONE'   	# Special value indicating "no register"



##### ALU Functions referenced explicitly                            #####

intsig ALUADD	'A_ADD'		# ALU should add its arguments



##### Possible instruction status values                             #####

intsig SAOK	'STAT_AOK'		# Normal execution

intsig SADR	'STAT_ADR'	# Invalid memory address

intsig SINS	'STAT_INS'	# Invalid instruction

intsig SHLT	'STAT_HLT'	# Halt instruction encountered



##### Signals that can be referenced by control logic ####################



##### Fetch stage inputs		#####

intsig pc 'pc'				# Program counter

##### Fetch stage computations		#####

intsig imem_icode 'imem_icode'		# icode field from instruction memory

intsig imem_ifun  'imem_ifun' 		# ifun field from instruction memory

intsig icode	  'icode'		# Instruction control code

intsig ifun	  'ifun'		# Instruction function

intsig rA	  'ra'			# rA field from instruction

intsig rB	  'rb'			# rB field from instruction

intsig valC	  'valc'		# Constant from instruction

intsig valP	  'valp'		# Address of following instruction

boolsig imem_error 'imem_error'		# Error signal from instruction memory

boolsig instr_valid 'instr_valid'	# Is fetched instruction valid?



##### Decode stage computations		#####

intsig valA	'vala'			# Value from register A port

intsig valB	'valb'			# Value from register B port



##### Execute stage computations	#####

intsig valE	'vale'			# Value computed by ALU

boolsig Cnd	'cond'			# Branch test



##### Memory stage computations		#####

intsig valM	'valm'			# Value read from memory

boolsig dmem_error 'dmem_error'		# Error signal from data memory





####################################################################

#    Control Signal Definitions.                                   #

####################################################################



################ Fetch Stage     ###################################



# Determine instruction code

int icode = [

	imem_error: INOP;

	1: imem_icode;		# Default: get from instruction memory

];



# Determine instruction function

int ifun = [

	imem_error: FNONE;

	1: imem_ifun;		# Default: get from instruction memory

];



bool instr_valid = icode in 

	{ INOP, IHALT, IRRMOVL, IIRMOVL, IRMMOVL, IMRMOVL,

	       IOPL, IJXX, ICALL, IRET, IPUSHL, IPOPL, IIADDL };



# Does fetched instruction require a regid byte?

bool need_regids =

	icode in { IRRMOVL, IOPL, IPUSHL, IPOPL, 

		     IIRMOVL, IRMMOVL, IMRMOVL, IIADDL };



# Does fetched instruction require a constant word?

bool need_valC =

	icode in { IIRMOVL, IRMMOVL, IMRMOVL, IJXX, ICALL, IIADDL };



################ Decode Stage    ###################################



## What register should be used as the A source?

int srcA = [

	icode in { IRRMOVL, IRMMOVL, IOPL, IPUSHL  } : rA;

	icode in { IPOPL, IRET } : RESP;

	1 : RNONE; # Don't need register

];



## What register should be used as the B source?

int srcB = [

	icode in { IOPL, IRMMOVL, IMRMOVL, IIADDL  } : rB;

	icode in { IPUSHL, IPOPL, ICALL, IRET } : RESP;

	1 : RNONE;  # Don't need register

];



## What register should be used as the E destination?

int dstE = [

	icode in { IRRMOVL } && Cnd : rB;

	icode in { IIRMOVL, IOPL, IIADDL } : rB;

	icode in { IPUSHL, IPOPL, ICALL, IRET } : RESP;

	1 : RNONE;  # Don't write any register

];



## What register should be used as the M destination?

int dstM = [

	icode in { IMRMOVL, IPOPL } : rA;

	1 : RNONE;  # Don't write any register

];



################ Execute Stage   ###################################



## Select input A to ALU

int aluA = [

	icode in { IRRMOVL, IOPL } : valA;

	icode in { IIRMOVL, IRMMOVL, IMRMOVL, IIADDL } : valC;

	icode in { ICALL, IPUSHL } : -4;

	icode in { IRET, IPOPL } : 4;

	# Other instructions don't need ALU

];



## Select input B to ALU

int aluB = [

	icode in { IRMMOVL, IMRMOVL, IOPL, ICALL, 

		      IPUSHL, IRET, IPOPL, IIADDL } : valB;

	icode in { IRRMOVL, IIRMOVL } : 0;

	# Other instructions don't need ALU

];



## Set the ALU function

int alufun = [

	icode == IOPL : ifun;

	1 : ALUADD;

];



## Should the condition codes be updated?

bool set_cc = icode in { IOPL, IIADDL };



################ Memory Stage    ###################################



## Set read control signal

bool mem_read = icode in { IMRMOVL, IPOPL, IRET };



## Set write control signal

bool mem_write = icode in { IRMMOVL, IPUSHL, ICALL };



## Select memory address

int mem_addr = [

	icode in { IRMMOVL, IPUSHL, ICALL, IMRMOVL } : valE;

	icode in { IPOPL, IRET } : valA;

	# Other instructions don't need address

];



## Select memory input data

int mem_data = [

	# Value from register

	icode in { IRMMOVL, IPUSHL } : valA;

	# Return PC

	icode == ICALL : valP;

	# Default: Don't write anything

];



## Determine instruction status

int Stat = [

	imem_error || dmem_error : SADR;

	!instr_valid: SINS;

	icode == IHALT : SHLT;

	1 : SAOK;

];



################ Program Counter Update ############################



## What address should instruction be fetched at



int new_pc = [

	# Call.  Use instruction constant

	icode == ICALL : valC;

	# Taken branch.  Use instruction constant

	icode == IJXX && Cnd : valC;

	# Completion of RET instruction.  Use value from stack

	icode == IRET : valM;

	# Default: Use incremented PC

	1 : valP;

];

#/* $end seq-all-hcl */
				\end{lstlisting}

  \end {CJK*}
\end {document}

