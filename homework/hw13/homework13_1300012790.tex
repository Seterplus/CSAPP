\documentclass {article}
\usepackage {geometry}
\usepackage {CJK}
\usepackage {amsmath}
\usepackage {amssymb}
\usepackage {indentfirst}
\usepackage {listings}
\usepackage {courier}

\lstset{basicstyle=\ttfamily,breaklines=true,numbers=left}

\begin{document}
  \begin {CJK*} {UTF8} {gbsn}
    \title {\textbf {\Huge Homework 13}}
		\author {郭天魁 \\ 信息科学技术学院 \\ 1300012790}

		\maketitle
		

		\section{Homework 13}
			\subsection{8.10}
				setjmp:		C. Called once, returns one or more times.

				longjmp:	B. Called once, never returns.

				execve:		B. Called once, never returns.

				fork:			A. Called once, returns twice.

			\subsection{8.23}
				第一个SIGUSR2被成功接收并sleep,此时第二个SIGUSR2到达,但信号处理程序还在处理第一个SIGUSR2,这导致其被阻塞,加入等待队列中。之后3个SIGUSR2均被舍弃。当第一个SIGUSR2的sleep结束后,父进程接收队列中的第二个SIGUSR2并处理,故总共只处理了两个SIGUSR2信号。

			\subsection{8.25}
				\begin{lstlisting}[language=C]
#include <stdio.h>
#include <unistd.h>
#include <signal.h>
#include <setjmp.h>

sigjmp_buf env;
void handler(int sig) {
	signal(SIGALRM, SIG_DFL);
	siglongjmp(env, 1);
}
char *tfgets(char *s, int size, FILE *stream) {
	signal(SIGALRM, handler);
	alarm(5);
	int val = sigsetjmp(env, 1);
	if (val)
		return NULL;
	else {
		char *r = fgets(s, size, stream);
		alarm(0);
		signal(SIGALRM, SIG_DFL);
		return r;
	}
}

char s[100];

int main() {
	char *p = tfgets(s, 100, stdin);
	puts(p ? p : "TIMEOUT");
	p = tfgets(s, 100, stdin);
	puts(p ? p : "TIMEOUT");
	return 0;
}
				\end{lstlisting}

  \end {CJK*}
\end {document}

