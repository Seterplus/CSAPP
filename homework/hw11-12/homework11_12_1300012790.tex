\documentclass {article}
\usepackage {geometry}
\usepackage {CJK}
\usepackage {amsmath}
\usepackage {amssymb}
\usepackage {indentfirst}
\usepackage {listings}
\usepackage {courier}

\begin{document}
  \begin {CJK*} {UTF8} {gbsn}
    \title {\textbf {\Huge Homework 11 \& Homework 12}}
		\author {郭天魁 \\ 信息科学技术学院 \\ 1300012790}

		\maketitle
		
		\section{Homework 11}
			\subsection{7.6}
				\begin{table}[h]
					\begin{tabular}{lcccc}
						Symbol & swap.o .symtab entry? & Symbol type & Module where defined & Section \\ \hline
						buf    & Y                     & extern      & main.o               & .data   \\
						bufp0  & Y                     & global      & swap.o               & .data   \\
						bufp1  & Y                     & local       & swap.o               & .bss    \\
						swap   & Y                     & global      & swap.o               & .text   \\
						temp   & N                     & -           & -                    & -       \\
						incr   & Y                     & local       & swap.o               & .text   \\
						count  & Y                     & local       & swap.o               & .bss   
					\end{tabular}
				\end{table}
				其中count初始化为0,经过测试的确在.bss中。
			
			\subsection{7.7}
				在bar5.c中的double x前加入static修饰符。
				
			\subsection{7.12}
				\begin{table}[h]
					\begin{tabular}{ccc}
						Line \# in Fig. 7.10 & Address    & Value      \\ \hline
						15                   & 0x080483CB & 0x0804945C \\
						16                   & 0x080483D0 & 0x08049458 \\
						18                   & 0x080483D8 & 0x08049548 \\
						18                   & 0x080483DC & 0x08049458 \\
						23                   & 0x080483E7 & 0x08049548
					\end{tabular}
				\end{table}
				
			\subsection{7.15}
				A. 分别为1561和460个。
				
				B. 会少.debug和.line部分,.strtab也可能不同。
				
				C. linux-vdso.so.1, libc.so.6, ld-linux-x86-64.so.2.

		\section{Homework 12}
			\subsection{8.9}
				\begin{table}[h]
					\begin{tabular}{cc}
						Process pair & Concurrent? \\ \hline
						AB           & N           \\
						AC           & Y           \\
						AD           & Y           \\
						BC           & Y           \\
						BD           & Y           \\
						CD           & Y          
					\end{tabular}
				\end{table}

			\subsection{8.12}
				8

			\subsection{8.18}
				ACE

			\subsection{8.19}
				$2^n$

  \end {CJK*}
\end {document}

