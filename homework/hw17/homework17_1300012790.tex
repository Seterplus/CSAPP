\documentclass {article}
\usepackage {geometry}
\usepackage {CJK}
\usepackage {amsmath}
\usepackage {amssymb}
\usepackage {indentfirst}
\usepackage {listings}
\usepackage {courier}

\lstset{basicstyle=\ttfamily,breaklines=true,numbers=left}

\begin{document}
  \begin {CJK*} {UTF8} {gbsn}
	\title {\textbf {\Huge Homework 17}}
		\author {郭天魁 \\ 信息科学技术学院 \\ 1300012790}

		\maketitle
		
		\section{Homework 17}
			\subsection{11.6}
				A. Tiny已经输出了header,只需要输出status line即可。
				
				\begin{lstlisting}[language=C]
void doit(int fd) 
{
	// ...

	/* Read request line and headers */
	Rio_readinitb(&rio, fd);
	Rio_readlineb(&rio, buf, MAXLINE);
	printf("%s", buf);
	sscanf(buf, "%s %s %s", method, uri, version);

	// ...
}
				\end{lstlisting}

				B. 使用Chromium访问了自带的home.html(http://localhost:6568/home.html),Tiny产生如下输出:

				\begin{lstlisting}[numbers=none]
GET /home.html HTTP/1.1
Host: localhost:6568
Connection: keep-alive
Cache-Control: max-age=0
Accept: text/html,application/xhtml+xml,application/xml;q=0.9,image/webp,*/*;q=0.8
User-Agent: Mozilla/5.0 (X11; Linux x86_64) AppleWebKit/537.36 (KHTML, like Gecko) Ubuntu Chromium/39.0.2171.65 Chrome/39.0.2171.65 Safari/537.36
Accept-Encoding: gzip, deflate, sdch
Accept-Language: zh-CN,zh;q=0.8,en;q=0.6,zh-TW;q=0.4

GET /godzilla.gif HTTP/1.1
Host: localhost:6568
Connection: keep-alive
Cache-Control: max-age=0
Accept: image/webp,*/*;q=0.8
User-Agent: Mozilla/5.0 (X11; Linux x86_64) AppleWebKit/537.36 (KHTML, like Gecko) Ubuntu Chromium/39.0.2171.65 Chrome/39.0.2171.65 Safari/537.36
Referer: http://localhost:6568/home.html
Accept-Encoding: gzip, deflate, sdch
Accept-Language: zh-CN,zh;q=0.8,en;q=0.6,zh-TW;q=0.4

				\end{lstlisting}

				C. HTTP/1.1。

				D.
				
				Host: 请求资源所在的域名/IP以及端口号。例如在Apache中会利用Host来连接对应的VirtualHost。本例中为Tiny运行的IP及端口。

				Connection: 指定完成本次连接后是否断开连接。本例中为保持连接。

				Cache-Control: 指定当前对象的一系列cacheable信息,如不缓存,Age阈值,freshness阈值等等。本例中为可缓存本次获取的对象,但不使用之前的缓存内容。

				Accept: 指定接受的介质类型。本例中为首选网页文件和图片文件。

				User-Agent: 浏览器信息。本例中为Ubuntu Chromium。

				Referer: 来源网页信息。本例中图片文件来源为home.html。

				Accept-Encoding: 指定接受的编码方法,通常为压缩方法。本例中为首选gzip压缩格式。

				Accept-Language: 指定接受的语言。本例中为首选中文。

			\subsection{11.7}
				\begin{lstlisting}[language=C]
int endswith(char *filename, char *ext) {
	char *dot = strrchr(filename, '.');
	return (dot && !strcmp(dot, ext));
}

int endswithany(char *filename, char *exts[], int n) {
	int i;
	for (i = 0; i < n; i++)
		if (endswith(filename, exts[i]))
			return 1;
	return 0;
}

/*
 * get_filetype - derive file type from file name
 */
void get_filetype(char *filename, char *filetype) 
{
	const char *mpg[] = {".mpg", ".mpeg", ".mp1", ".mp2", ".mp3", ".m1v", ".m1a", ".m2a", ".mpa", ".mpv"};
	const char *mp4[] = {".mp4", ".m4a", ".m4p", ".m4b", ".m4r", ".m4v"};
	if (endswith(filename, ".html"))
		strcpy(filetype, "text/html");
	else if (endswith(filename, ".gif"))
		strcpy(filetype, "image/gif");
	else if (endswith(filename, ".jpg"))
		strcpy(filetype, "image/jpeg");
	else if (endswithany(filename, mpg, sizeof(mpg) / sizeof(char *)))
		strcpy(filetype, "video/mpeg");
	else if (endswithany(filename, mp4, sizeof(mp4) / sizeof(char *)))
		strcpy(filetype, "video/mp4");
	else
		strcpy(filetype, "text/plain");
}
				\end{lstlisting}

				另外,你需要将所有的Rio\_前缀替换为rio\_。


  \end {CJK*}
\end {document}

